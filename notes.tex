\documentclass[14pt]{article}
\usepackage{amsmath}
\usepackage{amssymb,times}
\def\MM#1{\boldsymbol{#1}}

\title{How we do REXI}
\author{J. Shipton and C. J. Cotter}

\begin{document}

We are solving a system of the form

\begin{equation}
U(t) = \mathcal{L}U.
\end{equation}

Using an exponential integrator the solution of this system is

\begin{equation}
U(t) = e^{(t-t_0)\mathcal{L}}U(t_0).
\end{equation}

We compute the matrix exponential using the rational approximation

\begin{equation}
  e^{\tau\mathcal{L}}U(t_0) \approx \Sigma_{n=-N}^N\beta_n(\tau\mathcal{L} + \alpha_n)^{-1}U(t_0),
\end{equation}

\noindent where $\tau = t-t_0$, $\alpha_n$, and $\beta_n$ are complex
valued coefficiants given in Haut 2015, and computed using a python
version of Martin Schreiber's code.

For each $n$ we solve

\begin{equation}\label{neqn}
(\tau\mathcal{L} + \alpha_n)U_n = \beta_nU(t_0).
\end{equation}

For the linear shallow water system with constant $f$ we have

\begin{equation}
  \mathcal{L}U =
  \begin{pmatrix}
    -f\MM{u}^\perp - g\nabla h \\
    -H\nabla\cdot\MM{u}
  \end{pmatrix},
\end{equation}

\noindent so equation \ref{neqn} becomes

\begin{align}
  \tau(-f\MM{u_n}^\perp - g\nabla h_n) + \alpha_n\MM{u_n} &= \beta_n\MM{u_0}\\
  \tau(-H\nabla\cdot\MM{u_n}) + \alpha_nh_n &= \beta_nh_0
\end{align}

The weak form is

\begin{align}
  -\tau f\langle\MM{w}, \MM{u_n}^\perp\rangle + \tau g\langle\nabla\cdot\MM{w}, h_n\rangle + \alpha_n\langle\MM{w}, \MM{u_n}\rangle &= \beta_n\langle\MM{w}, \MM{u_0}\rangle \label{weak u}\\
  -\tau H\langle\phi, \nabla\cdot\MM{u_n}\rangle + \alpha_n\langle\phi, h_n\rangle &= \beta_n\langle\phi, h_0\rangle. \label{weak h}
\end{align}

Although $U(t_0)$ and the solution $U(t)$ are real valued, the complex
valued coefficients require that we work with complex numbers. Since
Firedrake does not yet have this capability, we write out the real and
imaginary components ourselves:

\begin{align}
  \MM{w} &= \MM{w}^r + i\MM{w}^i, \\
  \MM{u_n} &= \MM{u}_n^r + i\MM{u}_n^i
\end{align}

\noindent and likewise for $\phi$ and $h$ and the coeffiecients
$\alpha_n$ and $\beta_n$. Writing this out and equating real parts of
equations \ref{weak u}-\ref{weak h} gives

\begin{align}
  \begin{split}
    -\tau f\big(\langle\MM{w}^r, \MM{u_n^r}^\perp\rangle - \langle\MM{w}^i, \MM{u_n^i}^\perp\rangle\big) \\
  + \tau g\big(\langle\nabla\cdot\MM{w}^r, h_n^r\rangle - \langle\nabla\cdot\MM{w}^i, h_n^i\rangle\big) \\
  + \alpha_n^r(\langle\MM{w}^r, \MM{u}_n^r\rangle - \langle\MM{w}^i, \MM{u}_n^i\rangle) \\
  - \alpha_n^i(\langle\MM{w}^r, \MM{u}_n^i\rangle + \langle\MM{w}^i, \MM{u}_n^r\rangle) &= \beta_n^r\langle\MM{w}^r, \MM{u}_0\rangle - \beta_n^i\langle\MM{w}^i, \MM{u}_0\rangle
  \end{split}\\
  \begin{split}
    -\tau H \big(\langle\phi^r, \nabla\cdot\MM{u}_n^r\rangle - \langle\phi^i, \nabla\cdot\MM{u}_n^i\rangle\big) \\
    +\alpha_n^r\big(\langle\phi^r, h_n^r\rangle - \langle\phi^i, h_n^i\rangle\big) \\
    -\alpha_n^i\big(\langle\phi^r, h_n^i,\rangle + \langle\phi^i, h_n^r\rangle\big) &= \beta_n^r\langle\phi^r, h_n^r\rangle - \beta_n^i\langle\phi^i, h_n^i\rangle
  \end{split}
\end{align}

\end{document}
